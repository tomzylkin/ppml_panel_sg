%% LyX 2.1.3 created this file.  For more info, see http://www.lyx.org/.
%% Do not edit unless you really know what you are doing.
\RequirePackage{fix-cm}
\documentclass[12pt]{article}
\usepackage{mathptmx}
\usepackage[T1]{fontenc}
\usepackage[latin9]{inputenc}
\usepackage{geometry}
\geometry{verbose,tmargin=1.5cm,bmargin=2cm,lmargin=1.25cm,rmargin=1.25cm}
\setlength{\parskip}{0bp}
\setlength{\parindent}{0pt}
\usepackage{verbatim}
\usepackage{amsmath}
\usepackage{amssymb}
\usepackage{setspace}
\usepackage[authoryear]{natbib}
\PassOptionsToPackage{normalem}{ulem}
\usepackage{ulem}
\setstretch{1.175}
\usepackage[unicode=true]
 {hyperref}
\usepackage{breakurl}

\makeatletter
%%%%%%%%%%%%%%%%%%%%%%%%%%%%%% Textclass specific LaTeX commands.
\newenvironment{lyxlist}[1]
{\begin{list}{}
{\settowidth{\labelwidth}{#1}
 \setlength{\leftmargin}{\labelwidth}
 \addtolength{\leftmargin}{\labelsep}
 \renewcommand{\makelabel}[1]{##1\hfil}}}
{\end{list}}

%%%%%%%%%%%%%%%%%%%%%%%%%%%%%% User specified LaTeX commands.

\usepackage{fancyhdr}

\usepackage{setspace}
\providecommand{\tabularnewline}{\\}
\usepackage{pgf}\usepackage{pgfarrows}\usepackage{pgfnodes}\usepackage{pgfautomata}\usepackage{pgfheaps}\usepackage{abstract}\usepackage{longtable}\usepackage{rotating}

\newtheorem{theorem}{Theorem}
\newtheorem{acknowledgement}[theorem]{Acknowledgement}
\newtheorem{algorithm}[theorem]{Algorithm}
\newtheorem{axiom}[theorem]{Axiom}
\newtheorem{case}[theorem]{Case}
\newtheorem{claim}[theorem]{Claim}
\newtheorem{conclusion}[theorem]{Conclusion}
\newtheorem{condition}[theorem]{Condition}
\newtheorem{conjecture}[theorem]{Conjecture}
\newtheorem{corollary}{Corollary}
\newtheorem{criterion}[theorem]{Criterion}
\newtheorem{definition}[theorem]{Definition}
\newtheorem{example}[theorem]{Example}
\newtheorem{exercise}[theorem]{Exercise}
\newtheorem{lemma}{Lemma}
\newtheorem{notation}[theorem]{Notation}
\newtheorem{problem}[theorem]{Problem}
\newtheorem{proposition}{Proposition}
\newtheorem{remark}[theorem]{Remark}
\newtheorem{solution}[theorem]{Solution}
\newtheorem{summary}[theorem]{Summary}
\newenvironment{proof}[1][Proof]{\textbf{#1.} }{\  \rule{0.5em}{0.5em}}



\usepackage{hyperref}

\definecolor{darkblue}{rgb}{0.00,0.15,0.7}
\hypersetup{pdfpagemode=UseNone,colorlinks,breaklinks,
            linkcolor=darkblue,urlcolor=darkblue,
            anchorcolor=black,citecolor=black,
            pdfstartview= XYZ null null 1.00,
            %pdfstartview= fit
            pdftoolbar= true,
            pdfcenterwindow=false,
           pdfpagemode=UseNone,colorlinks,breaklinks, 
           }

%%% set spaces between references (squeeze to save space!)
%\setlength{\bibsep}{0.05cm}

\makeatother

\begin{document}

\subsection*{Title}

\texttt{ppml\_panel\_sg} - Fast Poisson Pseudo-Maximum Likelihood
(PPML) regression for panel ``gravity'' models with time-varying
origin and destination fixed effects and time-invariant pair fixed
effects.


\subsection*{Syntax}

\texttt{ppml\_panel\_sg} \href{http://www.stata.com/help.cgi?depvar}{depvar}
{[}\href{http://www.stata.com/help.cgi?indepvars}{indepvars}{]} {[}\href{http://www.stata.com/help.cgi?if}{if}{]}
{[}\href{http://www.stata.com/help.cgi?in}{in}{]}, \uline{ex}porter(\emph{exp\_id})
\uline{im}porter(\emph{imp\_id}) \uline{y}ear(\emph{time\_id})
{[}\hyperref[sub:Main-Options]{options}{]}\vspace{0.375cm}


\emph{exp\_id}, \emph{imp\_id}, and \emph{time\_id }are variables
that respectively identify the origin, destination, and time period
associated with each observation.


\subsection*{Description}

\texttt{ppml\_panel\_sg} enables faster computation of the many fixed
effects required for panel PPML structural gravity estimation. In
particular, it addresses the large number of ``pair-wise'' FEs needed
to consistently identify the effects of time-varying trade policies
such as regional trade agreements (see, e.g., \citealp{baier_free_2007,dai_trade-diversion_2014}).
It also simultaneously absorbs the origin-by-time and destination-by-time
FEs implied by theory. \vspace{0.375cm}


Some options and features of interest:
\begin{enumerate}
\item Programmed to run in Mata, making it much faster than existing Stata
Poisson commands for estimating the effects of trade policies.
\item Can store the estimated fixed effects in Stata's memory (but as a
single column each, rather than as a large matrix with many zeroes)
\item In addition to the pair fixed effects, also readily allows for specifications
with additional pair-specific linear time trends.
\item All fixed effects are also allowed to vary by industry, in case you
wish to examine industry-level variation in trade flows.
\item Can be used with strictly cross-sectional regressions (i.e., without
pair fixed effects).
\item Performs \citet{santos_silva_existence_2010}'s recommended check
for possible non-existence of estimates.
\end{enumerate}

\subsection*{Main Options\label{sub:Main-Options}}
\begin{lyxlist}{00.00.0000000000}
\item [{\textbf{\uline{nopair}}}] Use origin-time and destination-time
fixed effects only (do not include pair fixed effects).
\item [{\textbf{\uline{trend}}}] Add linear, pair-specific time trends.
\item [{\textbf{\uline{sym}}\textbf{metric}}] Assume pair fixed effects
apply symmetrically to flows in both directions, as in, e.g., \citet{anderson_terms_2016}.
Time trends, if specified, will also be symmetric.
\item [{\textbf{\uline{ind}}\textbf{ustry}(\emph{ind\_id})}] \noindent \emph{ind\_id}
is a varname identifying the industry associated with each observation.
When an \emph{ind\_id} is provided, the origin-time fixed effects
become origin-industry-time effects, the destination-time fixed effects
become destination-industry-time effects, and pair-specific terms
become origin-destination-industry-specific.
\item [{\textbf{\uline{off}}\textbf{set(}\emph{varname}\textbf{)}}] Include
\emph{varname} as a regressor with coefficient constrained to be equal
to 1.
\item [{\textbf{\uline{tol}}\textbf{erance}(\href{http://www.stata.com/help.cgi?\#}{\#{}})}] The
default tolerance is 1e-12.
\item [{\textbf{\uline{max}}\textbf{iter}(\href{http://www.stata.com/help.cgi?\#}{\#{}})}] The
default maximum number of iterations is 10,000.
\item [{\textbf{\uline{clus}}\textbf{ter}(\emph{cluster\_id})}] Specifies
clustered standard errors, clustered by \emph{cluster\_id. }The default
is clustering by \emph{exp\_id-imp\_id}{[}-\emph{ind\_id}{]}, unless
\textbf{nopair }is enabled.\emph{ }
\item [{\textbf{\uline{ro}}\textbf{bust}}] Use robust standard errors.
This is the default if \textbf{nopair} is enabled.
\item [{\textbf{\uline{no}}\textbf{sterr}}] Do not compute standard
errors (saves time if you only care about point estimates).
\item [{\textbf{\uline{verb}}\textbf{ose}(\href{http://www.stata.com/help.cgi?\#}{\#{}})}] Show
iterative output for every \href{http://www.stata.com/help.cgi?\#}{\#{}}th
iteration. Default is 0 (no output).
\end{lyxlist}

\subsection*{Guessing and Storing Values}

These options allow you to store results for fixed effects in memory
as well as use information from memory to set up initial guesses.
You may also utilize the ``multilateral resistances'' of \citet{anderson_gravity_2003}.\vspace{-0.0875cm}

\begin{lyxlist}{000000000.00.0000}
\item [{\textbf{\uline{ols}}\textbf{guess}}] Use \href{[https://ideas.repec.org/c/boc/bocode/s457874.html||reghdfe]}{reghdfe}
to initialize guesses for coefficient values.
\item [{\textbf{guessB}(\emph{str})}] Supply the name of a row vector with
guesses for coefficient values.
\item [{\textbf{guessD}(\emph{varname})}] \noindent Guess initial values
for the (exponentiated) set of pair fixed effects. Default is all
1s.
\item [{\textbf{guessS}(\emph{varname})}] Guess initial values for the
(exponentiated) set of origin-time fixed effects. Default is the share
of \href{http://www.stata.com/help.cgi?depvar}{depvar}\emph{ }within
each\emph{ }{[}\emph{ind\_id}-{]}\emph{time-id} associated with each
\emph{exp-id.}
\item [{\textbf{guessM}(\emph{varname})}] Guess initial values for the
(exponentiated) set of destination-time fixed effects. Default is
the share of \href{http://www.stata.com/help.cgi?depvar}{depvar}\emph{
}within each\emph{ }{[}\emph{ind\_id}-{]}\emph{time-id} associated
with each \emph{imp-id.}
\item [{\textbf{guessO}(\emph{varname})}] Guess initial values for the
set of ``outward'' multilateral resistances. Default is all 1s.
Overrides \textbf{genS}.
\item [{\textbf{guessI}(\emph{varname})}] Guess initial values for the
set of ``inward'' multilateral resistances. Default is all 1s. Overrides
\textbf{genM}.
\item [{\textbf{guessTT}(\emph{varname})}] \noindent Guess initial values
for pair time trends. These are not exponentiated. Default is all
0s.
\end{lyxlist}
\textbf{genD}(\href{http://www.stata.com/help.cgi?newvar}{newvar}),
\textbf{genS}(\href{http://www.stata.com/help.cgi?newvar}{newvar}),
\textbf{genM}(\href{http://www.stata.com/help.cgi?newvar}{newvar}),
\textbf{genO}(\href{http://www.stata.com/help.cgi?newvar}{newvar}),\textbf{
genI}(\href{http://www.stata.com/help.cgi?newvar}{newvar}), and \textbf{genTT}(\href{http://www.stata.com/help.cgi?newvar}{newvar}):
These options store fixed effects and/or time trend parameters in
memory as new variables.


\subsection*{Background}

As a typical application, consider the following PPML regression:
\begin{align}
X_{ijt} & =\exp\left[\ln S_{it}+\ln M_{jt}+\ln D_{ij}+b\times RTA_{ijt}\right]+e_{ijt}.\label{eq:estimation}
\end{align}


$X_{ijt}$ are international trade flows. $i$, $j$, and $t$ are
indices for origin, destination, and time. The goal is to consistently
estimate the average effect of $RTA_{ijt}$, a dummy variable for
the presence of a regional trade agreement on trade flows, using a
``structural gravity'' specification. The origin-time and destination-time
fixed effects---$S_{it}$ and $M_{jt}$---ensure the theoretical restrictions
implied by structural gravity are satisfied. The pair fixed effect---$D_{ij}$---then
absorbs all time-invariant pair characteristics that may be correlated
with the likelihood of forming an RTA.\vspace{0.375cm}


\noindent Computationally, the biggest obstacle to estimating \eqref{eq:estimation}
is the pair fixed effect term $D_{ij}$. Because a unique $D_{ij}$
must be computed for each pair, the number of $D_{ij}$'s increases
rapidly with the number of locations. For a balanced international
trade data set with 75 countries trading with each other over 10 years
(not an especially large sample for trade data), there will be on
the order of $75^{2}=5,625$ pair fixed effects that must be computed.
In addition (ignoring collinearity), we will also require $75\times2\times10=1,500$
origin-time and destination-time effects. The total number of parameters
needed to estimate \eqref{eq:estimation} (around $7,000$) would
normally require a long computing time in Stata, likely several hours
at least. If we push the number of locations and/or years further,
we will quickly approach Stata's matsize limits, beyond which estimation
becomes infeasible.\vspace{0.375cm}


To date, this is the only available Stata command that will perform
``fast'' estimation of specifications such as \eqref{eq:estimation}
using PPML. It works by manipulating the first order conditions of
the Poisson to produce analytical expressions for each of the fixed
effects that can be computed via simple iteration. In this way, it
both adapts and extends existing procedures described in \citet{guimaraes_simple_2010}
and \citet{figueiredo_industry_2015} for estimating Poisson models
with high dimensional fixed effects. These works and others are recommended
below for further reading.%
\begin{comment}
Make some sort of reference to Poisson PML following the ``workhorse''
statement on JSS's website.
\end{comment}



\subsection*{Examples}

To perform a basic panel estimation such as \eqref{eq:estimation}:\vspace{0.25cm}


\texttt{ppml\_panel\_sg trade rta, ex(iso\_o) im(iso\_d) y(year)}\vspace{0.5cm}


To add pair-specific time trends, i.e.,
\begin{align}
X_{ijt} & =\exp\left[\ln S_{it}+\ln M_{jt}+\ln D_{ij}+a_{ij}\times t+b\times RTA_{ijt}\right]+e_{ijt},\label{eq:2}
\end{align}
you would input:\vspace{0.25cm}


\texttt{ppml\_panel\_sg trade rta, ex(iso\_o) im(iso\_d) y(year) trend}\vspace{0.5cm}


If you want your pair fixed effects to be symmetric (i.e., $D_{ij}=D_{ji}$),
the syntax is:\vspace{0.25cm}


\texttt{ppml\_panel\_sg trade rta, ex(iso\_o) im(iso\_d) y(year) sym}\vspace{0.5cm}


To estimate coefficients of more traditional, time-invariant gravity
variables, such as bilateral distance, use the \textbf{nopair }option:\vspace{0.25cm}


\texttt{ppml\_panel\_sg trade ln\_dist colony language contiguity
rta, }

\texttt{ex(iso\_o) im(iso\_d) y(year) nopair}\vspace{0.5cm}


Unlike the regressions with pair fixed effects, however, obtaining
estimates for time-invariant regressors may not be noticeably faster
than existing methods (e.g., \href{http://www.stata.com/help.cgi?glm}{glm},
\href{https://ideas.repec.org/c/boc/bocode/s458102.html}{ppml}) unless
the number of origin-time and destination-time effects is sufficiently
large. You may also exclude the year ID in this last specification
if your data includes only 1 year. 


\subsection*{Advisory}

This estimation command is strictly intended for settings where the
dependent variable is spatial flows from one set of locations to another
(such as international trade or migration flows). It is not a generalized
Poisson fixed effects command. For more general problems that require
Poisson estimation, you may try: \href{http://www.stata.com/help.cgi?poisson}{poisson},
\href{http://www.stata.com/help.cgi?glm}{glm}, \href{https://ideas.repec.org/c/boc/bocode/s458102.html}{ppml},
\href{http://www.stata.com/help.cgi?xtpoisson}{xtpoisson}, \href{http://people.bu.edu/tsimcoe/data.html}{xtpqml},
and/or \href{https://ideas.repec.org/c/boc/bocode/s457777.html}{poi2hdfe}.
For an OLS command that can compute similar ``gravity'' specifications
using OLS, I recommend \href{https://ideas.repec.org/c/boc/bocode/s457874.html}{reghdfe}.\vspace{0.375cm}


As noted above, a useful feature of this command is that it will automatically
drop any of your main covariates which do not satisfy the condition
for guaranteeing the existence of estimates described in \citet{santos_silva_existence_2010}.
This should ensure convergence in most cases. However, you may still
encounter convergence issues in cases when linear time trends are
specified and when the data contains many zeroes. Future versions
of this command will seek to address this latter issue.

\vspace{0.375cm}


This is version 1.0 of this command. If you believe you have found
an error that can be replicated, or have other suggestions for improvements,
please feel free to \href{mailto:tomzylkin@gmail.com}{contact me}.%
\begin{comment}
Some other things to keep in mind:
\end{comment}



\subsection*{Acknowledgements}

I have adapted parts of my code from several other related commands.
These include: \href{https://ideas.repec.org/c/boc/bocode/s457777.html}{poi2hdfe},
by Paulo Guimar\~aes, \href{https://sites.google.com/site/hiegravity/stata-programs}{SILS},
by Keith Head and Thierry Mayer, and \href{https://ideas.repec.org/c/boc/bocode/s457874.html}{reghdfe}
by Sergio Correia. I give the utmost credit to each of these authors
for creating these programs and making their code available.\vspace{0.375cm}


I also thank Yoto Yotov, Davin Chor, Sergio Correia, Paulo Guimar\~aes,
and Jo\~ao Santos Silva for kindly taking the time to offer feedback
and suggest improvements. 


\subsection*{Further Reading}
\begin{itemize}
\item Structural gravity: \citet{anderson_gravity_2003,head_gravity_2014}
\item On the use of PPML to estimate gravity equations: \citet{santos_silva_log_2006,santos_silva_further_2011,arvis_poisson_2013,egger_glm_2014,fally_structural_2015}
\item Possible non-existence of Poisson MLE estimates: \citet{santos_silva_existence_2010}
\item Consistently estimating the effects of trade policies: \citet{baier_free_2007,dai_trade-diversion_2014,anderson_terms_2016,piermartini_estimating_2016}
\item Estimating models with high dimensional fixed effects: \citet{guimaraes_simple_2010,gaure_ols_2011,figueiredo_industry_2015,correia_reghdfe:_2016}
\item Multi-variate Steffenson's acceleration (used to accelerate convergence):
\citet{nievergelt_aitkens_1991}
\end{itemize}
\bibliographystyle{tomzcustom}
\bibliography{references}
\begin{comment}
A note on multilateral resistances?
\end{comment}

\end{document}
